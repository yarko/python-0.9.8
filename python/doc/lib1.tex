%% Master: lib.tex
\chapter{Introduction}

The Python library consists of three parts, with different levels of
integration with the interpreter.
Closest to the interpreter are built-in types, exceptions and functions.
Next are built-in modules, which are written in \C{} and linked statically
with the interpreter.
Finally there are standard modules that are implemented entirely in
Python, but are always available.
For efficiency, some standard modules may become built-in modules in
future versions of the interpreter.
\indexii{built-in}{types}
\indexii{built-in}{exceptions}
\indexii{built-in}{functions}
\indexii{built-in}{modules}
\indexii{standard}{modules}
\indexii{\C{}}{language}

\chapter{Built-in Types, Exceptions and Functions}
\nodename{Built-in Objects}

Names for built-in exceptions and functions are found in a separate
symbol table.  This table is searched last, so local and global
user-defined names can override built-in names.  Built-in types have
no names but are created easily by constructing an object of the
desired type (e.g., using a literal) and applying the built-in
function \code{type()} to it.  They are described together here for
easy reference.%
\footnote{Some descriptions sorely lack explanations of the exceptions
	that may be raised --- this will be fixed in a future version of
	this document.}
\indexii{built-in}{types}
\indexii{built-in}{exceptions}
\indexii{built-in}{functions}
\index{symbol table}
\bifuncindex{type}

\section{Built-in Types}

The following sections describe the standard types that are built into
the interpreter.  These are the numeric types, sequence types, and
several others, including types themselves.  There is no explicit
Boolean type; use integers instead.
\indexii{built-in}{types}
\indexii{Boolean}{type}

Some operations are supported by several object types; in particular,
all objects can be compared, tested for truth value, and converted to
a string (with the \code{`{\rm \ldots}`} notation).  The latter conversion is
implicitly used when an object is written by the \code{print} statement.
\stindex{print}

\subsection{Truth Value Testing}

Any object can be tested for truth value, for use in an \code{if} or
\code{while} condition or as operand of the Boolean operations below.
The following values are false:
\stindex{if}
\stindex{while}
\indexii{truth}{value}
\indexii{Boolean}{operations}
\index{false}

\begin{itemize}
\renewcommand{\indexsubitem}{(Built-in object)}

\item	\code{None}
	\ttindex{None}

\item	zero of any numeric type, e.g., \code{0}, \code{0L}, \code{0.0}.

\item	any empty sequence, e.g., \code{''}, \code{()}, \code{[]}.

\item	any empty mapping, e.g., \code{\{\}}.

\end{itemize}

\emph{All} other values are true --- so objects of many types are
always true.
\index{true}

\subsection{Boolean Operations}

These are the Boolean operations:
\indexii{Boolean}{operations}

\begin{tableiii}{|c|l|c|}{code}{Operation}{Result}{Notes}
  \lineiii{\var{x} or \var{y}}{if \var{x} is false, then \var{y}, else \var{x}}{(1)}
  \lineiii{\var{x} and \var{y}}{if \var{x} is false, then \var{x}, else \var{y}}{(1)}
  \lineiii{not \var{x}}{if \var{x} is false, then \code{1}, else \code{0}}{}
\end{tableiii}
\opindex{and}
\opindex{or}
\opindex{not}

\noindent
Notes:

\begin{description}

\item[(1)]
These only evaluate their second argument if needed for their outcome.

\end{description}

\subsection{Comparisons}

Comparison operations are supported by all objects:

\begin{tableiii}{|c|l|c|}{code}{Operation}{Meaning}{Notes}
  \lineiii{<}{strictly less than}{}
  \lineiii{<=}{less than or equal}{}
  \lineiii{>}{strictly greater than}{}
  \lineiii{>=}{greater than or equal}{}
  \lineiii{==}{equal}{}
  \lineiii{<>}{not equal}{(1)}
  \lineiii{!=}{not equal}{(1)}
  \lineiii{is}{object identity}{}
  \lineiii{is not}{negated object identity}{}
\end{tableiii}
\indexii{operator}{comparison}
\opindex{==} % XXX *All* others have funny characters < ! >
\opindex{is}
\opindex{is not}

\noindent
Notes:

\begin{description}

\item[(1)]
\code{<>} and \code{!=} are alternate spellings for the same operator.
(I couldn't choose between \ABC{} and \C{}! :-)
\indexii{\ABC{}}{language}
\indexii{\C{}}{language}

\end{description}

Objects of different types, except different numeric types, never
compare equal; such objects are ordered consistently but arbitrarily
(so that sorting a heterogeneous array yields a consistent result).
Furthermore, some types (e.g., windows) support only a degenerate
notion of comparison where any two objects of that type are unequal.
Again, such objects are ordered arbitrarily but consistently.
\indexii{types}{numeric}
\indexii{objects}{comparing}

(Implementation note: objects of different types except numbers are
ordered by their type names; objects of the same types that don't
support proper comparison are ordered by their address.)

Two more operations with the same syntactic priority, \code{in} and
\code{not in}, are supported only by sequence types (below).
\opindex{in}
\opindex{not in}

\subsection{Numeric Types}

There are three numeric types: \dfn{plain integers}, \dfn{long integers}, and
\dfn{floating point numbers}.  Plain integers (also just called \dfn{integers})
are implemented using \code{long} in \C{}, which gives them at least 32
bits of precision.  Long integers have unlimited precision.  Floating
point numbers are implemented using \code{double} in \C{}.  All bets on
their precision are off unless you happen to know the machine you are
working with.
\indexii{numeric}{types}
\indexii{integer}{types}
\indexii{integer}{type}
\indexiii{long}{integer}{type}
\indexii{floating point}{type}
\indexii{\C{}}{language}

Numbers are created by numeric literals or as the result of built-in
functions and operators.  Unadorned integer literals (including hex
and octal numbers) yield plain integers.  Integer literals with an \samp{L}
or \samp{l} suffix yield long integers
(\samp{L} is preferred because \code{1l} looks too much like eleven!).
Numeric literals containing a decimal point or an exponent sign yield
floating point numbers.
\indexii{numeric}{literals}
\indexii{integer}{literals}
\indexiii{long}{integer}{literals}
\indexii{floating point}{literals}
\indexii{hexadecimal}{literals}
\indexii{octal}{literals}

Python fully supports mixed arithmetic: when an binary arithmetic
operator has operands of different numeric types, the operand with the
``smaller'' type is converted to that of the other, where plain
integer is smaller than long integer is smaller than floating point.
Comparisons between numbers of mixed type use the same rule.%
\footnote{As a consequence, the list \code{[1, 2]} is considered equal
	to \code{[1.0, 2.0]}, and similar for tuples.}
The functions \code{int()}, \code{long()} and \code{float()} can be used
to coerce numbers to a specific type.
\index{arithmetic}
\bifuncindex{int}
\bifuncindex{long}
\bifuncindex{float}

All numeric types support the following operations:

\begin{tableiii}{|c|l|c|}{code}{Operation}{Result}{Notes}
  \lineiii{abs(\var{x})}{absolute value of \var{x}}{}
  \lineiii{int(\var{x})}{\var{x} converted to integer}{(1)}
  \lineiii{long(\var{x})}{\var{x} converted to long integer}{(1)}
  \lineiii{float(\var{x})}{\var{x} converted to floating point}{}
  \lineiii{-\var{x}}{\var{x} negated}{}
  \lineiii{+\var{x}}{\var{x} unchanged}{}
  \lineiii{\var{x} + \var{y}}{sum of \var{x} and \var{y}}{}
  \lineiii{\var{x} - \var{y}}{difference of \var{x} and \var{y}}{}
  \lineiii{\var{x} * \var{y}}{product of \var{x} and \var{y}}{}
  \lineiii{\var{x} / \var{y}}{quotient of \var{x} and \var{y}}{(2)}
  \lineiii{\var{x} \%{} \var{y}}{remainder of \code{\var{x} / \var{y}}}{}
  \lineiii{divmod(\var{x}, \var{y})}{the pair \code{(\var{x} / \var{y}, \var{x} \%{} \var{y})}}{(3)}
  \lineiii{pow(\var{x}, \var{y})}{\var{x} to the power \var{y}}{}
\end{tableiii}
\indexiii{operations on}{numeric}{types}

\noindent
Notes:
\begin{description}
\item[(1)]
Conversion from floating point to (long or plain) integer may round or
% XXXJH xref here
truncate as in \C{}; see functions \code{floor} and \code{ceil} in module
\code{math} for well-defined conversions.
\indexii{numeric}{conversions}
\bimodindex{math}
\indexii{\C{}}{language}

\item[(2)]
For (plain or long) integer division, the result is an integer; it
always truncates towards zero.
% XXXJH integer division is better defined nowadays
\indexii{integer}{division}
\indexiii{long}{integer}{division}

\item[(3)]
See the section on built-in functions for an exact definition.

\end{description}
% XXXJH exceptions: overflow (when? what operations?) zerodivision

\subsubsection{Bit-string Operations on Integer Types.}

Plain and long integer types support additional operations that make
sense only for bit-strings.  Negative numbers are treated as their 2's
complement value:

\begin{tableiii}{|c|l|c|}{code}{Operation}{Result}{Notes}
  \lineiii{\~\var{x}}{the bits of \var{x} inverted}{}
  \lineiii{\var{x} \^{} \var{y}}{bitwise \dfn{exclusive or} of \var{x} and \var{y}}{}
  \lineiii{\var{x} \&{} \var{y}}{bitwise \dfn{and} of \var{x} and \var{y}}{}
  \lineiii{\var{x} | \var{y}}{bitwise \dfn{or} of \var{x} and \var{y}}{}
  \lineiii{\var{x} << \var{n}}{\var{x} shifted left by \var{n} bits}{}
  \lineiii{\var{x} >> \var{n}}{\var{x} shifted right by \var{n} bits}{}
\end{tableiii}
% XXXJH what's `left'? `right'? maybe better use lsb or msb or something
\indexiii{operations on}{integer}{types}
\indexii{bit-string}{operations}
\indexii{shifting}{operations}
\indexii{masking}{operations}

\subsection{Sequence Types}

There are three sequence types: strings, lists and tuples.
Strings literals are written in single quotes: \code{'xyzzy'}.
Lists are constructed with square brackets,
separating items with commas:
\code{[a, b, c]}.
Tuples are constructed by the comma operator
(not within square brackets), with or without enclosing parentheses,
but an empty tuple must have the enclosing parentheses, e.g.,
\code{a, b, c} or \code{()}.  A single item tuple must have a trailing comma,
e.g., \code{(d,)}.
\indexii{sequence}{types}
\indexii{string}{type}
\indexii{tuple}{type}
\indexii{list}{type}

Sequence types support the following operations (\var{s} and \var{t} are
sequences of the same type; \var{n}, \var{i} and \var{j} are integers):

\begin{tableiii}{|c|l|c|}{code}{Operation}{Result}{Notes}
  \lineiii{len(\var{s})}{length of \var{s}}{}
  \lineiii{min(\var{s})}{smallest item of \var{s}}{}
  \lineiii{max(\var{s})}{largest item of \var{s}}{}
  \lineiii{\var{x} in \var{s}}{\code{1} if an item of \var{s} is equal to \var{x}, else \code{0}}{}
  \lineiii{\var{x} not in \var{s}}{\code{0} if an item of \var{s} is equal to \var{x}, else \code{1}}{}
  \lineiii{\var{s} + \var{t}}{the concatenation of \var{s} and \var{t}}{}
  \lineiii{\var{s} * \var{n}{\rm ,} \var{n} * \var{s}}{\var{n} copies of \var{s} concatenated}{}
  \lineiii{\var{s}[\var{i}]}{\var{i}'th item of \var{s}, origin 0}{(1)}
  \lineiii{\var{s}[\var{i}:\var{j}]}{slice of \var{s} from \var{i} to \var{j}}{(1), (2)}
\end{tableiii}
\indexiii{operations on}{sequence}{types}
\bifuncindex{len}
\bifuncindex{min}
\bifuncindex{max}
\indexii{concatenation}{operation}
\indexii{repetition}{operation}
\indexii{subscript}{operation}
\indexii{slice}{operation}
\opindex{in}
\opindex{not in}

\noindent
Notes:

% XXXJH all TeX-math expressions replaced by python-syntax expressions
\begin{description}
  
\item[(1)] If \var{i} or \var{j} is negative, the index is relative to
  the end of the string, i.e., \code{len(\var{s}) + \var{i}} or
  \code{len(\var{s}) + \var{j}} is substituted.  But note that \code{-0} is
  still \code{0}.
  
\item[(2)] The slice of \var{s} from \var{i} to \var{j} is defined as
  the sequence of items with index \var{k} such that \code{\var{i} <=
  \var{k} < \var{j}}.  If \var{i} or \var{j} is greater than
  \code{len(\var{s})}, use \code{len(\var{s})}.  If \var{i} is omitted,
  use \code{0}.  If \var{j} is omitted, use \code{len(\var{s})}.  If
  \var{i} is greater than or equal to \var{j}, the slice is empty.

\end{description}

\subsubsection{Mutable Sequence Types.}

List objects support additional operations that allow in-place
modification of the object.
These operations would be supported by other mutable sequence types
(when added to the language) as well.
Strings and tuples are immutable sequence types and such objects cannot
be modified once created.
The following operations are defined on mutable sequence types (where
\var{x} is an arbitrary object):
\indexiii{mutable}{sequence}{types}
\indexii{list}{type}

\begin{tableiii}{|c|l|c|}{code}{Operation}{Result}{Notes}
  \lineiii{\var{s}[\var{i}] = \var{x}}
	{item \var{i} of \var{s} is replaced by \var{x}}{}
  \lineiii{\var{s}[\var{i}:\var{j}] = \var{t}}
  	{slice of \var{s} from \var{i} to \var{j} is replaced by \var{t}}{}
  \lineiii{del \var{s}[\var{i}:\var{j}]}
	{same as \code{\var{s}[\var{i}:\var{j}] = []}}{}
  \lineiii{\var{s}.append(\var{x})}
	{same as \code{\var{s}[len(\var{x}):len(\var{x})] = [\var{x}]}}{}
  \lineiii{\var{s}.count(\var{x})}
	{return number of \var{i}'s for which \code{\var{s}[\var{i}] == \var{x}}}{}
  \lineiii{\var{s}.index(\var{x})}
	{return smallest \var{i} such that \code{\var{s}[\var{i}] == \var{x}}}{(1)}
  \lineiii{\var{s}.insert(\var{i}, \var{x})}
	{same as \code{\var{s}[\var{i}:\var{i}] = [\var{x}]}}{}
  \lineiii{\var{s}.remove(\var{x})}
	{same as \code{del \var{s}[\var{s}.index(\var{x})]}}{(1)}
  \lineiii{\var{s}.reverse()}
	{reverses the items of \var{s} in place}{}
  \lineiii{\var{s}.sort()}
	{permutes the items of \var{s} to satisfy
        \code{\var{s}[\var{i}] <= \var{s}[\var{j}]},
        for \code{\var{i} < \var{j}}}{(2)}
\end{tableiii}
\indexiv{operations on}{mutable}{sequence}{types}
\indexiii{operations on}{sequence}{types}
\indexiii{operations on}{list}{type}
\indexii{subscript}{assignment}
\indexii{slice}{assignment}
\stindex{del}
\renewcommand{\indexsubitem}{(list method)}
\ttindex{append}
\ttindex{count}
\ttindex{index}
\ttindex{insert}
\ttindex{remove}
\ttindex{reverse}
\ttindex{sort}

\noindent
Notes:
\begin{description}
\item[(1)] Raises an exception when \var{x} is not found in \var{s}.
  
\item[(2)] The \code{sort()} method takes an optional argument
  specifying a comparison function of two arguments (list items) which
  should return \code{-1}, \code{0} or \code{1} depending on whether the
  first argument is considered smaller than, equal to, or larger than the
  second argument.  Note that this slows the sorting process down
  considerably; e.g. to sort an array in reverse order it is much faster
  to use calls to \code{sort()} and \code{reverse()} than to use
  \code{sort()} with a comparison function that reverses the ordering of
  the elements.
\end{description}

\subsection{Mapping Types}

A \dfn{mapping} object maps values of one type (the key type) to
arbitrary objects.  Mappings are mutable objects.  There is currently
only one mapping type, the \dfn{dictionary}.  A dictionary's keys are
strings.
\indexii{mapping}{types}
\indexii{dictionary}{type}

Dictionaries are created by placing a comma-separated list of
\code{\var{key}: \var{value}} pairs within braces, for example:
\code{\{'jack': 4098, 'sape': 4127\}}.

The following operations are defined on mappings (where \var{a} is a
mapping, \var{k} is a key and \var{x} is an arbitrary object):

\begin{tableiii}{|c|l|c|}{code}{Operation}{Result}{Notes}
  \lineiii{len(\var{a})}{the number of items in \var{a}}{}
  \lineiii{\var{a}[\var{k}]}{the item of \var{a} with key \var{k}}{(1)}
  \lineiii{\var{a}[\var{k}] = \var{x}}{set \code{\var{a}[\var{k}]} to \var{x}}{}
  \lineiii{del \var{a}[\var{k}]}{remove \code{\var{a}[\var{k}]} from \var{a}}{(1)}
  \lineiii{\var{a}.keys()}{a copy of \var{a}'s list of keys}{(2)}
  \lineiii{\var{a}.has_key(\var{k})}{true if \var{a} has a key \var{k}}{}
\end{tableiii}
\indexiii{operations on}{mapping}{types}
\indexiii{operations on}{dictionary}{type}
\stindex{del}
\bifuncindex{len}
\renewcommand{\indexsubitem}{(dictionary method)}
\ttindex{keys}
\ttindex{has_key}

% XXXJH some lines above, you talk about `true', elsewhere you
% explicitely states \code{0} or \code{1}.
\noindent
Notes:
\begin{description}
\item[(1)] Raises an exception if \var{k} is not in the map.

\item[(2)] Keys are listed in random order.
\end{description}

\subsection{Other Built-in Types}

The interpreter supports several other kinds of objects.
Most of these support only one or two operations.

\subsubsection{Modules.}

The only special operation on a module is attribute access:
\code{\var{m}.\var{name}}, where \var{m} is a module and \var{name} accesses
a name defined in \var{m}'s symbol table.  Module attributes can be
assigned to.  (Note that the \code{import} statement is not, strictly
spoken, an operation on a module object; \code{import \var{foo}} does not
require a module object named \var{foo} to exist, rather it requires
an (external) \emph{definition} for a module named \var{foo}
somewhere.)

A special member of every module is \code{__dict__}.
This is the dictionary containing the module's symbol table.
Modifying this dictionary will actually change the module's symbol
table, but direct assignment to the \code{__dict__} attribute is not
possible (i.e., you can write \code{\var{m}.__dict__['a'] = 1}, which
defines \code{\var{m}.a} to be \code{1}, but you can't write \code{\var{m}.__dict__ = \{\}}.

Modules are written like this: \code{<module 'sys'>}.

\subsubsection{Classes and Class Instances.}
% XXXJH cross ref here
(See the Python Reference Manual for these.)

\subsubsection{Functions.}

Function objects are created by function definitions.  The only
operation on a function object is to call it:
\code{\var{func}(\var{argument-list})}.

There are really two flavors of function objects: built-in functions
and user-defined functions.  Both support the same operation (to call
the function), but the implementation is different, hence the
different object types.

The implementation adds two special read-only attributes:
\code{\var{f}.func_code} is a function's \dfn{code object} (see below) and
\code{\var{f}.func_globals} is the dictionary used as the function's
global name space (this is the same as \code{\var{m}.__dict__} where
\var{m} is the module in which the function \var{f} was defined).

\subsubsection{Methods.}

Methods are functions that are called using the attribute notation.
There are two flavors: built-in methods (such as \code{append()} on
lists) and class instance methods.  Built-in methods are described
with the types that support them.

The implementation adds two special read-only attributes to class
instance methods: \code{\var{m}.im_self} is the object whose method this
is, and \code{\var{m}.im_func} is the function implementing the method.
Calling \code{\var{m}(\var{arg-1}, \var{arg-2}, {\rm \ldots},
\var{arg-n})} is completely equivalent to calling
\code{\var{m}.im_func(\var{m}.im_self, \var{arg-1}, \var{arg-2}, {\rm
\ldots}, \var{arg-n})}.

(See the Python Reference Manual for more info.)

\subsubsection{Type Objects.}

Type objects represent the various object types.  An object's type is
% XXXJH xref here
accessed by the built-in function \code{type()}.  There are no special
operations on types.

Types are written like this: \code{<type 'int'>}.

\subsubsection{The Null Object.}

This object is returned by functions that don't explicitly return a
value.  It supports no special operations.  There is exactly one null
object, named \code{None} (a built-in name).

It is written as \code{None}.

\subsubsection{File Objects.}

File objects are implemented using \C{}'s \code{stdio} package and can be
% XXXJH xref here
created with the built-in function \code{open()} described under
Built-in Functions below.

When a file operation fails for an I/O-related reason, the exception
\code{IOError} is raised.  This includes situations where the
operation is not defined for some reason, like \code{seek()} on a tty
device or writing a file opened for reading.

Files have the following methods:


\renewcommand{\indexsubitem}{(file method)}

\begin{funcdesc}{close}{}
  Close the file.  A closed file cannot be read or written anymore.
\end{funcdesc}

\begin{funcdesc}{flush}{}
  Flush the internal buffer, like \code{stdio}'s \code{fflush()}.
\end{funcdesc}

\begin{funcdesc}{isatty}{}
  Return \code{1} if the file is connected to a tty(-like) device, else
  \code{0}.
\end{funcdesc}

\begin{funcdesc}{read}{size}
  Read at most \var{size} bytes from the file (less if the read hits
  \EOF{} or no more data is immediately available on a pipe, tty or
  similar device).  If the \var{size} argument is omitted, read all
  data until \EOF{} is reached.  The bytes are returned as a string
  object.  An empty string is returned when \EOF{} is encountered
  immediately.  (For certain files, like ttys, it makes sense to
  continue reading after an \EOF{} is hit.)
\end{funcdesc}

\begin{funcdesc}{readline}{}
  Read one entire line from the file.  A trailing newline character is
  kept in the string (but may be absent when a file ends with an
  incomplete line).  An empty string is returned when \EOF{} is hit
  immediately.  Note: unlike \code{stdio}'s \code{fgets()}, the returned
  string contains null characters (\code{'\e 0'}) if they occurred in the
  input.
\end{funcdesc}

\begin{funcdesc}{readlines}{}
  Read until \EOF{} using \code{readline()} and return a list containing
  the lines thus read.
\end{funcdesc}

\begin{funcdesc}{seek}{offset\, whence}
  Set the file's current position, like \code{stdio}'s \code{fseek()}.
  The \var{whence} argument is optional and defaults to \code{0}
  (absolute file positioning); other values are \code{1} (seek
  relative to the current position) and \code{2} (seek relative to the
  file's end).  There is no return value.
\end{funcdesc}

\begin{funcdesc}{tell}{}
  Return the file's current position, like \code{stdio}'s \code{ftell()}.
\end{funcdesc}

\begin{funcdesc}{write}{str}
  Write a string to the file.  There is no return value.
\end{funcdesc}

\subsubsection{Internal Objects.}

(See the Python Reference Manual for these.)

\subsection{Special Attributes}

The implementation adds a few special read-only attributes to several
object types, where they are relevant:

\begin{itemize}

\item
\code{\var{x}.__dict__} is a dictionary of some sort used to store an
object's (writable) attributes;

\item
\code{\var{x}.__methods__} lists the methods of many built-in object types,
e.g., \code{[].__methods__} is
% XXXJH results in?, yields?, written down as an example
\code{['append', 'count', 'index', 'insert', 'remove', 'reverse', 'sort']};

\item
\code{\var{x}.__members__} lists data attributes;

\item
\code{\var{x}.__class__} is the class to which a class instance belongs;

\item
\code{\var{x}.__bases__} is the tuple of base classes of a class object.

\end{itemize}

\section{Built-in Exceptions}

Exceptions are string objects.  Two distinct string objects with the
same value are different exceptions.  This is done to force programmers
to use exception names rather than their string value when specifying
exception handlers.  The string value of all built-in exceptions is
their name, but this is not a requirement for user-defined exceptions
or exceptions defined by library modules.

The following exceptions can be generated by the interpreter or
built-in functions.  Except where mentioned, they have an `associated
value' indicating the detailed cause of the error.  This may be a
string or a tuple containing several items of information (e.g., an
error code and a string explaining the code).

User code can raise built-in exceptions.  This can be used to test an
exception handler or to report an error condition `just like' the
situation in which the interpreter raises the same exception; but
beware that there is nothing to prevent user code from raising an
inappropriate error.

\renewcommand{\indexsubitem}{(built-in exception)}

\begin{excdesc}{AttributeError}
% xref to attribute reference?
  Raised when an attribute reference or assignment fails.  (When an
  object does not support attributes references or attribute assignments
  at all, \code{TypeError} is raised.)
\end{excdesc}

\begin{excdesc}{EOFError}
% XXXJH xrefs here
  Raised when one of the built-in functions (\code{input()} or
  \code{raw_input()}) hits an end-of-file condition (\EOF{}) without
  reading any data.
% XXXJH xrefs here
  (N.B.: the \code{read()} and \code{readline()} methods of file
  objects return an empty string when they hit \EOF{}.)  No associated value.
\end{excdesc}

\begin{excdesc}{IOError}
% XXXJH xrefs here
  Raised when an I/O operation (such as a \code{print} statement, the
  built-in \code{open()} function or a method of a file object) fails
  for an I/O-related reason, e.g., `file not found', `disk full'.
\end{excdesc}

\begin{excdesc}{ImportError}
% XXXJH xref to import statement?
  Raised when an \code{import} statement fails to find the module
  definition or when a \code{from {\rm \ldots} import} fails to find a
  name that is to be imported.
\end{excdesc}

\begin{excdesc}{IndexError}
% XXXJH xref to sequences
  Raised when a sequence subscript is out of range.  (Slice indices are
  silently truncated to fall in the allowed range; if an index is not a
  plain integer, \code{TypeError} is raised.)
\end{excdesc}

\begin{excdesc}{KeyError}
% XXXJH xref to mapping objects?
  Raised when a mapping (dictionary) key is not found in the set of
  existing keys.
\end{excdesc}

\begin{excdesc}{KeyboardInterrupt}
  Raised when the user hits the interrupt key (normally
  \kbd{Control-C} or
\key{DEL}).  During execution, a check for interrupts is made regularly.
% XXXJH xrefs here
  Interrupts typed when a built-in function \code{input()} or
  \code{raw_input()}) is waiting for input also raise this exception.  No
  associated value.
\end{excdesc}

\begin{excdesc}{MemoryError}
  Raised when an operation runs out of memory but the situation may
  still be rescued (by deleting some objects).  The associated value is
  a string indicating what kind of (internal) operation ran out of memory.
  Note that because of the underlying memory management architecture
  (\C{}'s \code{malloc()} function), the interpreter may not always be able
  to completely recover from this situation; it nevertheless raises an
  exception so that a stack traceback can be printed, in case a run-away
  program was the cause.
\end{excdesc}

\begin{excdesc}{NameError}
  Raised when a local or global name is not found.  This applies only
  to unqualified names.  The associated value is the name that could
  not be found.
\end{excdesc}

\begin{excdesc}{OverflowError}
% XXXJH reference to long's and/or int's?
  Raised when the result of an arithmetic operation is too large to be
  represented.  This cannot occur for long integers (which would rather
  raise \code{MemoryError} than give up).  Because of the lack of
  standardization of floating point exception handling in \C{}, most
  floating point operations also aren't checked.  For plain integers,
  all operations that can overflow are checked except left shift, where
  typical applications prefer to drop bits than raise an exception.
\end{excdesc}

\begin{excdesc}{RuntimeError}
  Raised when an error is detected that doesn't fall in any of the
  other categories.  The associated value is a string indicating what
  precisely went wrong.  (This exception is a relic from a previous
  version of the interpreter; it is not used any more except by some
  extension modules that haven't been converted to define their own
  exceptions yet.)
\end{excdesc}

\begin{excdesc}{SyntaxError}
% XXXJH xref to these functions?
  Raised when the parser encounters a syntax error.  This may occur in
  an \code{import} statement, in a call to the built-in functions
  \code{eval()}, \code{exec()}, \code{execfile()} or \code{input()}, or
  when reading the initial script or standard input (also
  interactively).
\end{excdesc}

\begin{excdesc}{SystemError}
  Raised when the interpreter finds an internal error, but the
  situation does not look so serious to cause it to abandon all hope.
  The associated value is a string indicating what went wrong (in
  low-level terms).
  
  You should report this to the author or maintainer of your Python
  interpreter.  Be sure to report the version string of the Python
  interpreter (\code{sys.version}; it is also printed at the start of an
  interactive Python session), the exact error message (the exception's
  associated value) and if possible the source of the program that
  triggered the error.
\end{excdesc}

\begin{excdesc}{SystemExit}
% XXXJH xref to module sys?
  This exception is raised by the \code{sys.exit()} function.  When it
  is not handled, the Python interpreter exits; no stack traceback is
  printed.  If the associated value is a plain integer, it specifies the
  system exit status (passed to \C{}'s \code{exit()} function); if it is
  \code{None}, the exit status is zero; if it has another type (such as
  a string), the object's value is printed and the exit status is one.
  
  A call to \code{sys.exit} is translated into an exception so that
  clean-up handlers (\code{finally} clauses of \code{try} statements)
  can be executed, and so that a debugger can execute a script without
  running the risk of losing control.  The \code{posix._exit()} function
  can be used if it is absolutely positively necessary to exit
  immediately (e.g., after a \code{fork()} in the child process).
\end{excdesc}

\begin{excdesc}{TypeError}
  Raised when a built-in operation or function is applied to an object
  of inappropriate type.  The associated value is a string giving
  details about the type mismatch.
\end{excdesc}

\begin{excdesc}{ValueError}
  Raised when a built-in operation or function receives an argument
  that has the right type but an inappropriate value, and the
  situation is not described by a more precise exception such as
  \code{IndexError}.
\end{excdesc}

\begin{excdesc}{ZeroDivisionError}
  Raised when the second argument of a division or modulo operation is
  zero.  The associated value is a string indicating the type of the
  operands and the operation.
\end{excdesc}

\section{Built-in Functions}

The Python interpreter has a number of functions built into it that
are always available.  They are listed here in alphabetical order.


\renewcommand{\indexsubitem}{(built-in function)}
\begin{funcdesc}{abs}{x}
  Return the absolute value of a number.  The argument may be a plain
  or long integer or a floating point number.
\end{funcdesc}

\begin{funcdesc}{apply}{func\, args}
% XXXJH better: \var{func} must be.... and: the second --> \var{args}
  The first argument must be a callable object (a user-defined or
  built-in function or method, or a class object).  The second argument
  must be a tuple, possibly empty or a singleton.  The function is
  called with the tuple as argument list; the number of arguments is the
  same as the length of the tuple.  (This is different from just calling
  \code{\var{func}(\var{args})}, since in that case there is always
  exactly one argument.)
\end{funcdesc}

\begin{funcdesc}{chr}{i}
  Return a string of one character whose \ASCII{} code is the integer
  \var{i}, e.g., \code{chr(97)} returns the string \code{'a'}.  This is the
  inverse of \code{ord()}.  The argument must be in the range [0..255],
  inclusive.
\end{funcdesc}

\begin{funcdesc}{cmp}{x\, y}
  Compare the two objects \var{x} and \var{y} and return an integer
  according to the outcome.  The return value is negative if \code{\var{x}
  < \var{y}}, zero if \code{\var{x} == \var{y}} and strictly positive if
  \code{\var{x} > \var{y}}.
\end{funcdesc}

\begin{funcdesc}{coerce}{x\, y}
  Return a tuple consisting of the two numeric arguments converted to
  a common type, using the same rules as used by arithmetic
  operations.
\end{funcdesc}

\begin{funcdesc}{dir}{}
  Without arguments, return the list of names in the current local
  symbol table.  With a module, class or class instance object as
  argument (or anything else that has a \code{__dict__} attribute),
  returns the list of names in that object's attribute dictionary.
  The resulting list is sorted.  For example:

\bcode\begin{verbatim}
>>> import sys
>>> dir()
['sys']
>>> dir(sys)
['argv', 'exit', 'modules', 'path', 'stderr', 'stdin', 'stdout']
>>> 
\end{verbatim}\ecode
\end{funcdesc}

\begin{funcdesc}{divmod}{a\, b}
  Take two numbers as arguments and return a pair of integers
  consisting of their integer quotient and remainder.  With mixed
  operand types, the rules for binary arithmetic operators apply.  For
  plain and long integers, the result is the same as
  \code{(\var{a} / \var{b}, \var{a} \%{} \var{b})}.
  For floating point numbers the result is the same as
  \code{(math.floor(\var{a} / \var{b}), \var{a} \%{} \var{b})}.
\end{funcdesc}

\begin{funcdesc}{eval}{s\, globals\, locals}
  The arguments are a string and two optional dictionaries.  The
  string argument is parsed and evaluated as a Python expression
  (technically speaking, a condition list) using the dictionaries as
  global and local name space.  The string must not begin with
  whitespace, nor must it contain null bytes.  The return value is the
  result of the expression.  If the third argument is omitted it
  defaults to the second.  If both dictionaries are omitted, the
  expression is executed in the environment where \code{eval} is
  called.  Syntax errors are reported as exceptions.  Example:

\bcode\begin{verbatim}
>>> x = 1
>>> print eval('x+1')
2
>>> 
\end{verbatim}\ecode
\end{funcdesc}

\begin{funcdesc}{exec}{s\, globals\, locals}
  Similar to \code{eval}, but parses and executes the string as a
  sequence of statements.  The return value is \code{None}.  The string
  must not begin with whitespace and must end with a newline
  (\code{'\e n'}).
  Multiple lines separated by newlines are accepted; the
  normal indentation rules must be obeyed.  Syntax errors are reported
  as exceptions.  Example:

\bcode\begin{verbatim}
>>> x = 1
>>> exec('x = x+1\n')
>>> print x
2
>>> 
\end{verbatim}\ecode
\end{funcdesc}

\begin{funcdesc}{execfile}{filename\, globals\, locals}
  Similar to \code{exec}, but opens and parses a file instead of
  taking
  its input from a string.
\end{funcdesc}

\begin{funcdesc}{float}{x}
  Convert a number to floating point.  The argument may be a plain or
  long integer or a floating point number.
\end{funcdesc}

\begin{funcdesc}{getattr}{object\, name}
  The arguments are an object and a string.  The string must be the
  name
  of one of the object's attributes.  The result is the value of that
  attribute.  For example, \code{getattr(\var{x}, '\var{foobar}')} is equivalent to
  \code{\var{x}.\var{foobar}}.
\end{funcdesc}

\begin{funcdesc}{hex}{x}
  Convert a number to a hexadecimal string.  The result is a valid
  Python expression.
\end{funcdesc}

\begin{funcdesc}{input}{prompt}
  Almost equivalent to \code{eval(raw_input(\var{prompt}))}.  As for
  \code{raw_input()}, the prompt argument is optional.  The difference is
  that a long input expression may be broken over multiple lines using the
  backslash convention.
\end{funcdesc}

\begin{funcdesc}{int}{x}
  Convert a number to a plain integer.  The argument may be a plain or
  long integer or a floating point number.
\end{funcdesc}

\begin{funcdesc}{len}{s}
  Return the length (the number of items) of an object.  The argument
% XXXJH xrefs to sequence and/or mapping?
  may be a sequence (string, tuple or list) or a mapping (dictionary).
\end{funcdesc}

\begin{funcdesc}{long}{x}
  Convert a number to a long integer.  The argument may be a plain or
  long integer or a floating point number.
\end{funcdesc}

\begin{funcdesc}{max}{s}
  Return the largest item of a non-empty sequence (string, tuple or
  list).
\end{funcdesc}

\begin{funcdesc}{min}{s}
  Return the smallest item of a non-empty sequence (string, tuple or
  list).
\end{funcdesc}

\begin{funcdesc}{oct}{x}
  Convert a number to an octal string.  The result is a valid Python
  expression.
\end{funcdesc}

\begin{funcdesc}{open}{filename\, mode}
  % XXXJH xrefs here to Built-in types?
  Return a new file object (described earlier under Built-in Types).
  The string arguments are the same as for \code{stdio}'s
  \code{fopen()}: \var{filename} is the file name to be opened,
  \var{mode} indicates how the file is to be opened: \code{'r'} for
  reading, \code{'w'} for writing (truncating an existing file), and
  \code{'a'} opens it for appending.  Modes \code{'r+'}, \code{'w+'} and
  \code{'a+'} open the file for updating, provided the underlying
  \code{stdio} library understands this.  On systems that differentiate
  between binary and text files, \code{'b'} appended to the mode opens
  the file in binary mode.  If the file cannot be opened, \code{IOError}
  is raised.
\end{funcdesc}

\begin{funcdesc}{ord}{c}
  Return the \ASCII{} value of a string of one character.  E.g.,
  \code{ord('a')} returns the integer \code{97}.  This is the inverse of
  \code{chr()}.
\end{funcdesc}

\begin{funcdesc}{pow}{x\, y}
  Return \var{x} to the power \var{y}.  The arguments must have
  numeric types.  With mixed operand types, the rules for binary
  arithmetic operators apply.  The effective operand type is also the
  type of the result; if the result is not expressible in this type, the
  function raises an exception; e.g., \code{pow(2, -1)} is not allowed.
\end{funcdesc}

\begin{funcdesc}{range}{start\, end\, step}
  This is a versatile function to create lists containing arithmetic
  progressions.  It is most often used in \code{for} loops.  The
  arguments must be plain integers.  If the \var{step} argument is
  omitted, it defaults to \code{1}.  If the \var{start} argument is
  omitted, it defaults to \code{0}.  The full form returns a list of
  plain integers \code{[\var{start}, \var{start} + \var{step},
  \var{start} + 2 * \var{step}, \ldots]}.  If \var{step} is positive,
  the last element is the largest \code{\var{start} + \var{i} *
  \var{step}} less than \var{end}; if \var{step} is negative, the last
  element is the largest \code{\var{start} + \var{i} * \var{step}}
  greater than \var{end}.  \var{step} must not be zero.  Example:

\bcode\begin{verbatim}
>>> range(10)
[0, 1, 2, 3, 4, 5, 6, 7, 8, 9]
>>> range(1, 11)
[1, 2, 3, 4, 5, 6, 7, 8, 9, 10]
>>> range(0, 30, 5)
[0, 5, 10, 15, 20, 25]
>>> range(0, 10, 3)
[0, 3, 6, 9]
>>> range(0, -10, -1)
[0, -1, -2, -3, -4, -5, -6, -7, -8, -9]
>>> range(0)
[]
>>> range(1, 0)
[]
>>> 
\end{verbatim}\ecode
\end{funcdesc}

\begin{funcdesc}{raw_input}{prompt}
  The string argument is optional; if present, it is written to
  standard
  output without a trailing newline.  The function then reads a line
  from input, converts it to a string (stripping a trailing newline),
  and returns that.  When \EOF{} is read, \code{EOFError} is raised.
  Example:

\bcode\begin{verbatim}
>>> s = raw_input('--> ')
--> Monty Python's Flying Circus
>>> s
'Monty Python\'s Flying Circus'
>>> 
\end{verbatim}\ecode
\end{funcdesc}

\begin{funcdesc}{reload}{module}
  Re-parse and re-initialize an already imported \var{module}.  The
  argument must be a module object, so it must have been successfully
  imported before.  This is useful if you have edited the module source
  file using an external editor and want to try out the new version
  without leaving the Python interpreter.  Note that if a module is
  syntactically correct but its initialization fails, the first
  \code{import} statement for it does not import the name, but does
  create a (partially initialized) module object; to reload the module
  you must first \code{import} it again (this will just make the
  partially initialized module object available) before you can
  \code{reload()} it.
\end{funcdesc}

\begin{funcdesc}{repr}{object}
  This function returns exactly the same value as \code{`\var{object}`}.
  It is sometimes useful to be able to access this operation as an
  ordinary function.
\end{funcdesc}

\begin{funcdesc}{setattr}{object\, name\, value}
  This is the counterpart of \code{getattr}.  The arguments are an
  object, a string and an arbitrary value.  The string must be the name
  of one of the object's attributes.  The function assigns the value to
  the attribute, provided the object allows it.  For example,
  \code{setattr(\var{x}, '\var{foobar}', 123)} is equivalent to
  \code{\var{x}.\var{foobar} = 123}.
\end{funcdesc}

\begin{funcdesc}{str}{object}
  This function returns \code{repr(\var{object})} unless \var{object}
  is a string, in which case it returns \var{object} unchanged.
  It is sometimes useful to make sure that a value is a string without
  surrounding it with string quotes like \code{repr(\var{object})}
  does if its argument is a string.
\end{funcdesc}

\begin{funcdesc}{type}{object}
% XXXJH xref to buil-in objects here?
  Return the type of an \var{object}.  The return value is a type
  object.  There is not much you can do with type objects except compare
  them to other type objects; e.g., the following checks if a variable
  is a string:

\bcode\begin{verbatim}
>>> if type(x) == type(''): print 'It is a string'
\end{verbatim}\ecode
\end{funcdesc}
