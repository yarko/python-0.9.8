%% Master: lib.tex
\chapter{Built-in Modules}

The modules described in this section are built into the interpreter.
They must be imported using \code{import}.
Some modules are not always available; it is a configuration option to
provide them.
Details are listed with the descriptions, but the best way to see if
a module exists in a particular implementation is to attempt to import
it.

\section{Built-in Module \sectcode{sys}}

\bimodindex{sys}
This module provides access to some variables used or maintained by the
interpreter and to functions that interact strongly with the interpreter.
It is always available.

\renewcommand{\indexsubitem}{(in module sys)}
\begin{datadesc}{argv}
  The list of command line arguments passed to a Python script.
  \code{sys.argv[0]} is the script name.
  If no script name was passed to the Python interpreter,
  \code{sys.argv} is empty.
\end{datadesc}

\begin{datadesc}{exc_type}
\dataline{exc_value}
\dataline{exc_traceback}
  These three variables are not always defined; they are set when an
  exception handler (an \code{except} clause of a \code{try} statement) is
  invoked.  Their meaning is: \code{exc_type} gets the exception type of
  the exception being handled; \code{exc_value} gets the exception
  parameter (its \dfn{associated value} or the second argument to
  \code{raise}); \code{exc_traceback} gets a traceback object which
  encapsulates the call stack at the point where the exception
  originally occurred.
\end{datadesc}

\begin{funcdesc}{exit}{n}
  Exit from Python with numeric exit status \var{n}.  This is
  implemented by raising the \code{SystemExit} exception, so cleanup
  actions specified by \code{finally} clauses of \code{try} statements
  are honored, and it is possible to catch the exit attempt at an outer
  level.
\end{funcdesc}

\begin{datadesc}{exitfunc}
  This value is not actually defined by the module, but can be set by
  the user (or by a program) to specify a clean-up action at program
  exit.  When set, it should be a parameterless function.  This function
  will be called when the interpreter exits in any way (but not when a
  fatal error occurs: in that case the interpreter's internal state
  cannot be trusted).
\end{datadesc}

\begin{datadesc}{last_type}
\dataline{last_value}
\dataline{last_traceback}
  These three variables are not always defined; they are set when an
  exception is not handled and the interpreter prints an error message
  and a stack traceback.  Their intended use is to allow an interactive
  user to import a debugger module and engage in post-mortem debugging
  without having to re-execute the command that cause the error (which
  may be hard to reproduce).  The meaning of the variables is the same
  as that of \code{exc_type}, \code{exc_value} and \code{exc_tracaback},
  respectively.
\end{datadesc}

\begin{datadesc}{modules}
  Gives the list of modules that have already been loaded.
  This can be manipulated to force reloading of modules and other tricks.
\end{datadesc}

\begin{datadesc}{path}
  A list of strings that specifies the search path for modules.
  Initialized from the environment variable \code{PYTHONPATH}, or an
  installation-dependent default.
\end{datadesc}

\begin{datadesc}{ps1}
\dataline{ps2}
  Strings specifying the primary and secondary prompt of the
  interpreter.  These are only defined if the interpreter is in
  interactive mode.  Their initial values in this case are
  \code{'>>> '} and \code{'... '}.
\end{datadesc}

\begin{funcdesc}{settrace}{tracefunc}
  Set the system's trace function, which allows you to implement a
  Python source code debugger in Python.  The standard modules
  \code{pdb} and \code{wdb} are such debuggers; the difference is that
  \code{wdb} uses windows and needs STDWIN, while \code{pdb} has a
  line-oriented interface not unlike dbx.  See the file \file{pdb.doc}
  in the Python library source directory for more documentation (both
  about \code{pdb} and \code{sys.trace}).
\end{funcdesc}
\stmodindex{pdb}
\stmodindex{wdb}
\index{trace function}

\begin{funcdesc}{setprofile}{profilefunc}
  Set the system's profile function, which allows you to implement a
  Python source code profiler in Python.  The system's profile function
  is called similarly to the system's trace function (see
  \code{sys.settrace}), but it isn't called for each executed line of
  code (only on call and return and when an exception occurs).  Also,
  its return value is not used, so it can just return \code{None}.
\end{funcdesc}
\index{profile function}

\begin{datadesc}{stdin}
\dataline{stdout}
\dataline{stderr}
  File objects corresponding to the interpreter's standard input,
  output and error streams.  \code{sys.stdin} is used for all
  interpreter input except for scripts but including calls to
  \code{input()} and \code{raw_input()}.  \code{sys.stdout} is used
  for the output of \code{print} and expression statements and for the
  prompts of \code{input()} and \code{raw_input()}.  The interpreter's
  own prompts and its error messages are written to stderr.  Assigning
  to \code{sys.stderr} has no effect on the interpreter; it can be
  used to write error messages to stderr using \code{print}.
%JHXXX is this still correct??
\end{datadesc}

\section{Built-in Module \sectcode{__main__}}

\bimodindex{__main__}
This module represents the (otherwise anonymous) scope in which the
interpreter's main program executes --- commands read either from
standard input or from a script file.

\section{Built-in Module \sectcode{math}}

\bimodindex{math}
\renewcommand{\indexsubitem}{(in module math)}
This module is always available.
It provides access to the mathematical functions defined by the C
standard.
They are:
\iftexi
\begin{funcdesc}{acos}{x}
\funcline{asin}{x}
\funcline{atan}{x}
\funcline{atan2}{x, y}
\funcline{ceil}{x}
\funcline{cos}{x}
\funcline{cosh}{x}
\funcline{exp}{x}
\funcline{fabs}{x}
\funcline{floor}{x}
\funcline{fmod}{x, y}
\funcline{frexp}{x}
\funcline{ldexp}{x, y}
\funcline{log}{x}
\funcline{log10}{x}
\funcline{modf}{x}
\funcline{pow}{x, y}
\funcline{sin}{x}
\funcline{sinh}{x}
\funcline{sqrt}{x}
\funcline{tan}{x}
\funcline{tanh}{x}
\end{funcdesc}
\else
\code{acos(\varvars{x})},
\code{asin(\varvars{x})},
\code{atan(\varvars{x})},
\code{atan2(\varvars{x\, y})},
\code{ceil(\varvars{x})},
\code{cos(\varvars{x})},
\code{cosh(\varvars{x})},
\code{exp(\varvars{x})},
\code{fabs(\varvars{x})},
\code{floor(\varvars{x})},
\code{fmod(\varvars{x\, y})},
\code{frexp(\varvars{x})},
\code{ldexp(\varvars{x\, y})},
\code{log(\varvars{x})},
\code{log10(\varvars{x})},
\code{modf(\varvars{x})},
\code{pow(\varvars{x\, y})},
\code{sin(\varvars{x})},
\code{sinh(\varvars{x})},
\code{sqrt(\varvars{x})},
\code{tan(\varvars{x})},
\code{tanh(\varvars{x})}.
\fi

Note that \code{frexp} and \code{modf} have a different call/return
pattern than their C equivalents: they take a single argument and
return a pair of values, rather than returning their second return
value through an `output parameter' (there is no such thing in Python).

The module also defines two mathematical constants:
\iftexi
\begin{datadesc}{pi}
\dataline{e}
\end{datadesc}
\else
\code{pi} and \code{e}.
\fi

\section{Built-in Module \sectcode{time}}

\bimodindex{time}
This module provides various time-related functions.
It is always available.
Functions are:

\renewcommand{\indexsubitem}{(in module time)}
\begin{funcdesc}{sleep}{secs}
Suspend execution for the given number of seconds.  The argument may
be a floating point number to indicate a more precise sleep time; the
precision obtainable depends on the accuracy of the system clock but
is usually in the order of 1/100th or 1/60th of a second.
\end{funcdesc}

\begin{funcdesc}{time}{}
Return the ``wall clock time'' as a floating point number expressed in
seconds since the ``Epoch'' (Thursday January 1, 00:00:00, 1970 UCT on
\UNIX{} machines).  Note that even though the time is always returned
as a floating point number, not all systems provide wall clock time
with a better precision than 1 second.  An alternative for measuring
precise intervals is \code{millitimer}, described below.
\end{funcdesc}

\noindent
In most versions the following functions also exist:

\begin{funcdesc}{millisleep}{msecs}
Suspend execution for the given number of milliseconds.  (Obsolete,
you can now use use \code{sleep} with a floating point argument.)
\end{funcdesc}

\begin{funcdesc}{millitimer}{}
  Return the number of milliseconds of real time elapsed since some
  point in the past that is fixed per execution of the python
  interpreter (but may change in each following run).  The return
  value may be negative, and it may wrap around.
\end{funcdesc}

\noindent
The granularity of the milliseconds functions may be more than a
millisecond (100 msecs on Amoeba, 1/60 sec on the Mac).

\section{Built-in Module \sectcode{regex}}

\bimodindex{regex}
This module provides regular expression matching operations similar to
those found in Emacs.  It is always available.

By default the patterns are Emacs-style regular expressions; there is
a way to change the syntax to match that of several well-known
\UNIX{} utilities.

This module is 8-bit clean: both patterns and strings may contain null
bytes and characters whose high bit is set.

\strong{Please note:} There is a little-known fact about Python string literals
which means that you don't usually have to worry about doubling
backslashes, even though they are used to escape special characters in
string literals as well as in regular expressions.  This is because
Python doesn't remove backslashes from string literals if they are
followed by an unrecognized escape character.  \emph{However}, if you
want to include a literal \dfn{backslash} in a regular expression
represented as a string literal, you have to \emph{quadruple} it.  E.g.
to extract LaTeX \samp{\e section\{{\rm \ldots}\}} headers from a document, you can
use this pattern: \code{'\e \e \e\e section\{\e (.*\e )\}'}.

The module defines these functions, and an exception:

\renewcommand{\indexsubitem}{(in module regex)}
\begin{funcdesc}{match}{pattern\, string}
  Return how many characters at the beginning of \var{string} match
  the regular expression \var{pattern}.  Return \code{-1} if the
  string does not match the pattern (this is different from a
  zero-length match!).
\end{funcdesc}

\begin{funcdesc}{search}{pattern\, string}
  Return the first position in \var{string} that matches the regular
  expression \var{pattern}.  Return -1 if no position in the string
  matches the pattern (this is different from a zero-length match
  anywhere!).
\end{funcdesc}

\begin{funcdesc}{compile}{pattern}
  Compile a regular expression pattern into a regular expression
  object, which can be used for matching using its \code{match} and
  \code{search} methods, described below.  The sequence

\bcode\begin{verbatim}
prog = regex.compile(pat)
result = prog.match(str)
\end{verbatim}\ecode

is equivalent to

\bcode\begin{verbatim}
result = regex.match(pat, str)
\end{verbatim}\ecode

but the version using \code{compile()} is more efficient when multiple
regular expressions are used concurrently in a single program.  (The
compiled version of the last pattern passed to \code{regex.match()} or
\code{regex.search()} is cached, so programs that use only a single
regular expression at a time needn't worry about compiling regular
expressions.)
\end{funcdesc}

\begin{funcdesc}{set_syntax}{flags}
  Set the syntax to be used by future calls to \code{compile},
  \code{match} and \code{search}.  (Already compiled expression objects
  are not affected.)  The argument is an integer which is the OR of
  several flag bits.  The return value is the previous value of
  the syntax flags.  Names for the flags are defined in the standard
  module \code{regex_syntax}; read the file \file{regex_syntax.py} for
  more information.
\end{funcdesc}

\begin{excdesc}{error}
  Exception raised when a string passed to one of the functions here
  is not a valid regular expression (e.g., unmatched parentheses) or
  when some other error occurs during compilation or matching.  (It is
  never an error if a string contains no match for a pattern.)
\end{excdesc}

\noindent
Compiled regular expression objects support these methods:

\renewcommand{\indexsubitem}{(regex method)}
\begin{funcdesc}{match}{string\, pos}
  Return how many characters at the beginning of \var{string} match
  the compiled regular expression.  Return \code{-1} if the string
  does not match the pattern (this is different from a zero-length
  match!).
  
  The optional second parameter \var{pos} gives an index in the string
  where the search is to start; it defaults to \code{0}.  This is not
  completely equivalent to slicing the string; the \code{'\^'} pattern
  character matches at the real begin of the string and at positions
  just after a newline, not necessarily at the index where the search
  is to start.
%JHXXX added \var{pos} in description
\end{funcdesc}

\begin{funcdesc}{search}{string\, pos}
  Return the first position in \var{string} that matches the regular
  expression \code{pattern}.  Return \code{-1} if no position in the
  string matches the pattern (this is different from a zero-length
  match anywhere!).
  
  The optional second parameter has the same meaning as for the
  \code{match} method.
\end{funcdesc}

\noindent
Compiled regular expressions support one data attribute:

\renewcommand{\indexsubitem}{(regex attribute)}
\begin{datadesc}{regs}
  This attribute is only valid when the last call to the \code{match}
  or \code{search} method found a match.  Its value is a tuple of
  pairs of indices corresponding to the beginning and end of all
  parenthesized groups in the pattern.  Indices are relative to the
  string argument passed to \code{match} or \code{search}.  The 0-th
  tuple gives the beginning and end or the whole pattern.
\end{datadesc}

\section{Built-in Module \sectcode{marshal}}

\bimodindex{marshal}
This module contains two functions that can read and write Python
values in a binary format.  The format is specific to Python, but
independent of machine architecture issues (e.g., you can write a
Python value to a file on a VAX, transport the file to a Mac, and read
it back there).  Details of the format not explained here; read the
source if you're interested.

Not all Python object types are supported; in general, only objects
whose value is independent from a particular invocation of Python can
be written and read by this module.  The following types are supported:
\code{None}, integers, long integers, floating point numbers,
strings, tuples, lists, dictionaries, and code objects, where it
should be understood that tuples, lists and dictionaries are only
supported as long as the values contained therein are themselves
supported; and recursive lists and dictionaries should not be written
(they will cause an infinite loop).

The module defines these functions:

\renewcommand{\indexsubitem}{(in module marshal)}
\begin{funcdesc}{dump}{value\, file}
  Write the value on the open file.  The value must be a supported
  type.  The file must be an open file object such as
  \code{sys.stdout} or returned by \code{open()} or
  \code{posix.popen()}.
  
  If the value has an unsupported type, garbage is written which cannot
  be read back by \code{load()}.
\end{funcdesc}

\begin{funcdesc}{load}{file}
  Read one value from the open file and return it.  If no valid value
  is read, raise \code{EOFError}, \code{ValueError} or
  \code{TypeError}.  The file must be an open file object.
\end{funcdesc}

\section{Built-in module \sectcode{struct}}
\indexii{C}{structures}

This module performs conversions between Python values and C
structs represented as Python strings.  It uses \dfn{format strings}
(explained below) as a compact descriptions of the lay-out of the C
structs and the intended conversion to/from Python values.

The module defines the following exception and functions:

\renewcommand{\indexsubitem}{(in module struct)}
\begin{excdesc}{error}
  Exception raised on various occasions; argument is a string
  describing what is wrong.
\end{excdesc}

\begin{funcdesc}{pack}{fmt\, v1\, v2\, {\rm \ldots}}
  Return a string containing the values
  \code{\var{v1}, \var{v2}, {\rm \ldots}} packed according to the given
  format.  The arguments must match the values required by the format
  exactly.
\end{funcdesc}

\begin{funcdesc}{unpack}{fmt\, string}
  Unpack the string (presumably packed by \code{pack(\var{fmt}, {\rm \ldots})})
  according to the given format.  The result is a tuple even if it
  contains exactly one item.  The string must contain exactly the
  amount of data required by the format (i.e.  \code{len(\var{string})} must
  equal \code{calcsize(\var{fmt})}).
\end{funcdesc}

\begin{funcdesc}{calcsize}{fmt}
  Return the size of the struct (and hence of the string)
  corresponding to the given format.
\end{funcdesc}

Format characters have the following meaning; the conversion between C
and Python values should be obvious given their types:

\begin{tableiii}{|c|l|l|}{samp}{Format}{C}{Python}
  \lineiii{x}{pad byte}{no value}
  \lineiii{c}{char}{string of length 1}
  \lineiii{b}{signed char}{integer}
  \lineiii{h}{short}{integer}
  \lineiii{i}{int}{integer}
  \lineiii{l}{long}{integer}
  \lineiii{f}{float}{float}
  \lineiii{d}{double}{float}
\end{tableiii}

A format character may be preceded by an integral repeat count; e.g.
the format string \code{'4h'} means exactly the same as \code{'hhhh'}.

C numbers are represented in the machine's native format and byte
order, and properly aligned by skipping pad bytes if necessary
(according to the rules used by the C compiler).

Examples (all on a big-endian machine):

\bcode\begin{verbatim}
pack('hhl', 1, 2, 3) == '\000\001\000\002\000\000\000\003'
unpack('hhl', '\000\001\000\002\000\000\000\003') == (1, 2, 3)
calcsize('hhl') == 8
\end{verbatim}\ecode

Hint: to align the end of a structure to the alignment requirement of
a particular type, end the format with the code for that type with a
repeat count of zero, e.g. the format \code{'llh0l'} specifies two
pad bytes at the end, assuming longs are aligned on 4-byte boundaries.

(More format characters are planned, e.g. \code{'s'} for character
arrays, upper case for unsigned variants, and a way to specify the
byte order, which is useful for [de]constructing network packets and
reading/writing portable binary file formats like TIFF and AIFF.)
