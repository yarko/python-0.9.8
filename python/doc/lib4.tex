%% Master: lib.tex
\chapter{MOST OPERATING SYSTEMS}

\section{Built-in Module \sectcode{posix}}

\bimodindex{posix}

This module provides access to operating system functionality that is
standardized by the C Standard and the POSIX standard (a thinly diguised
\UNIX{} interface).
It is available in all Python versions except on the Macintosh;
the MS-DOS version does not support certain functions.
The descriptions below are very terse; refer to the
corresponding \UNIX{} manual entry for more information.

Errors are reported as exceptions; the usual exceptions are given
for type errors, while errors reported by the system calls raise
\code{posix.error}, described below.

Module \code{posix} defines the following data items:

\renewcommand{\indexsubitem}{(data in module posix)}
\begin{datadesc}{environ}
A dictionary representing the string environment at the time
the interpreter was started.
(Modifying this dictionary does not affect the string environment of the
interpreter.)
For example,
\code{posix.environ['HOME']}
is the pathname of your home directory, equivalent to
\code{getenv("HOME")}
in C.
\end{datadesc}

\renewcommand{\indexsubitem}{(exception in module posix)}
\begin{excdesc}{error}
This exception is raised when an POSIX function returns a
POSIX-related error (e.g., not for illegal argument types).  Its
string value is \code{'posix.error'}.  The accompanying value is a
pair containing the numeric error code from \code{errno} and the
corresponding string, as would be printed by the C function
\code{perror()}.
\end{excdesc}

It defines the following functions:

\renewcommand{\indexsubitem}{(in module posix)}
\begin{funcdesc}{chdir}{path}
Change the current working directory to \var{path}.
\end{funcdesc}

\begin{funcdesc}{chmod}{path\, mode}
Change the mode of \var{path} to the numeric \var{mode}.
\end{funcdesc}

\begin{funcdesc}{_exit}{n}
Exit to the system with status \var{n}, without calling cleanup
handlers, flushing stdio buffers, etc.
(Not on MS-DOS.)

Note: the standard way to exit is \code{sys.exit(\var{n})}.
\code{posix.exit()} should normally only be used in the child process
after a \code{fork()}.
\end{funcdesc}

\begin{funcdesc}{exec}{path\, args}
Execute the executable \var{path} with argument list \var{args},
replacing the current process (i.e., the Python interpreter).
The argument list may be a tuple or list of strings.
(Not on MS-DOS.)
\end{funcdesc}

\begin{funcdesc}{fork}{}
Fork a child process.  Return 0 in the child, the child's process id
in the parent.
(Not on MS-DOS.)
\end{funcdesc}

\begin{funcdesc}{getcwd}{}
Return a string representing the current working directory.
\end{funcdesc}

\begin{funcdesc}{getegid}{}
Return the current process's effective group id.
(Not on MS-DOS.)
\end{funcdesc}

\begin{funcdesc}{geteuid}{}
Return the current process's effective user id.
(Not on MS-DOS.)
\end{funcdesc}

\begin{funcdesc}{getgid}{}
Return the current process's group id.
(Not on MS-DOS.)
\end{funcdesc}

\begin{funcdesc}{getpid}{}
Return the current process id.
(Not on MS-DOS.)
\end{funcdesc}

\begin{funcdesc}{getppid}{}
Return the parent's process id.
(Not on MS-DOS.)
\end{funcdesc}

\begin{funcdesc}{getuid}{}
Return the current process's user id.
(Not on MS-DOS.)
\end{funcdesc}

\begin{funcdesc}{kill}{pid\, sig}
Kill the process \var{pid} with signal \var{sig}.
(Not on MS-DOS.)
\end{funcdesc}

\begin{funcdesc}{link}{src\, dst}
Create a hard link pointing to \var{src} named \var{dst}.
(Not on MS-DOS.)
\end{funcdesc}

\begin{funcdesc}{listdir}{path}
Return a list containing the names of the entries in the directory.
The list is in arbitrary order.  It includes the special entries
\code{'.'} and \code{'..'} if they are present in the directory.
\end{funcdesc}

\begin{funcdesc}{lstat}{path}
Like \code{stat()}, but do not follow symbolic links.  (On systems
without symbolic links, this is identical to \code{posix.stat}.)
\end{funcdesc}

\begin{funcdesc}{mkdir}{path\, mode}
Create a directory named \var{path} with numeric mode \var{mode}.
\end{funcdesc}

\begin{funcdesc}{nice}{increment}
Add \var{incr} to the process' ``niceness''.  Return the new niceness.
(Not on MS-DOS.)
\end{funcdesc}

\begin{funcdesc}{popen}{command\, mode}
Open a pipe to or from \var{command}.  The return value is an open
file object connected to the pipe, which can be read or written
depending on whether \var{mode} is \code{'r'} or \code{'w'}.
(Not on MS-DOS.)
\end{funcdesc}

\begin{funcdesc}{readlink}{path}
Return a string representing the path to which the symbolic link
points.  (On systems without symbolic links, this always raises
\code{posix.error}.)
\end{funcdesc}

\begin{funcdesc}{rename}{src\, dst}
Rename the file or directory \var{src} to \var{dst}.
\end{funcdesc}

\begin{funcdesc}{rmdir}{path}
Remove the directory \var{path}.
\end{funcdesc}

\begin{funcdesc}{stat}{path}
Perform a {\em stat} system call on the given path.  The return value
is a tuple of at least 10 integers giving the most important (and
portable) members of the {\em stat} structure, in the order
\code{st_mode},
\code{st_ino},
\code{st_dev},
\code{st_nlink},
\code{st_uid},
\code{st_gid},
\code{st_size},
\code{st_atime},
\code{st_mtime},
\code{st_ctime}.
More items may be added at the end by some implementations.
(On MS-DOS, some items are filled with dummy values.)

Note: The standard module \code{stat} defines functions and constants
that are useful for extracting information from a stat structure.
\end{funcdesc}

\begin{funcdesc}{symlink}{src\, dst}
Create a symbolic link pointing to \var{src} named \var{dst}.  (On
systems without symbolic links, this always raises
\code{posix.error}.)
\end{funcdesc}

\begin{funcdesc}{system}{command}
Execute the command (a string) in a subshell.  This is implemented by
calling the Standard C function \code{system()}, and has the same
limitations.  Changes to \code{posix.environ}, \code{sys.stdin} etc. are
not reflected in the environment of the executed command.  The return
value is the exit status of the process as returned by Standard C
\code{system()}.
\end{funcdesc}

\begin{funcdesc}{times}{}
Return a 4-tuple of floating point numbers indicating accumulated CPU
times, in seconds.  The items are: user time, system time, children's
user time, and children's system time, in that order.  See the \UNIX{}
manual page {\it times}(2).  (Not on MS-DOS.)
\end{funcdesc}

\begin{funcdesc}{umask}{mask}
Set the current numeric umask and returns the previous umask.
(Not on MS-DOS.)
\end{funcdesc}

\begin{funcdesc}{uname}{}
Return a 5-tuple containing information identifying the current
operating system.  The tuple contains 5 strings:
\code{(\var{sysname}, \var{nodename}, \var{release}, \var{version}, \var{machine})}.
Some systems truncate the nodename to 8
characters or to the leading component; an better way to get the
hostname is \code{socket.gethostname()}.  (Not on MS-DOS, nor on older
\UNIX{} systems.)
\end{funcdesc}

\begin{funcdesc}{unlink}{path}
Unlink \var{path}.
\end{funcdesc}

\begin{funcdesc}{utime}{path\, \(atime\, mtime\)}
Set the access and modified time of the file to the given values.
(The second argument is a tuple of two items.)
\end{funcdesc}

\begin{funcdesc}{wait}{}
Wait for completion of a child process, and return a tuple containing
its pid and exit status indication (encoded as by \UNIX{}).
(Not on MS-DOS.)
\end{funcdesc}

\begin{funcdesc}{waitpid}{pid\, options}
Wait for completion of a child process given by proces id, and return
a tuple containing its pid and exit status indication (encoded as by
\UNIX{}).  The semantics of the call are affected by the value of
the integer options, which should be 0 for normal operation.  (If the
system does not support waitpid(), this always raises
\code{posix.error}.  Not on MS-DOS.)
\end{funcdesc}

\section{Standard Module \sectcode{posixpath}}

\stmodindex{posixpath}
This module implements some useful functions on POSIX pathnames.

\renewcommand{\indexsubitem}{(in module posixpath)}
\begin{funcdesc}{basename}{p}
Return the base name of pathname
\var{p}.
This is the second half of the pair returned by
\code{posixpath.split(\var{p})}.
\end{funcdesc}

\begin{funcdesc}{commonprefix}{list}
Return the longest string that is a prefix of all strings in
\var{list}.
If
\var{list}
is empty, return the empty string (\code{''}).
\end{funcdesc}

\begin{funcdesc}{exists}{p}
Return true if
\var{p}
refers to an existing path.
\end{funcdesc}

\begin{funcdesc}{expanduser}{p}
Return the argument with an initial component of \samp{\~} or
\samp{\~\var{user}} replaced by that \var{user}'s home directory.  An
initial \samp{\~{}} is replaced by the environment variable \code{\${}HOME};
an initial \samp{\~\var{user}} is looked up in the password directory through
the built-in module \code{pwd}.  If the expansion fails, or if the
path does not begin with a tilde, the path is returned unchanged.
\end{funcdesc}

\begin{funcdesc}{isabs}{p}
Return true if \var{p} is an absolute pathname (begins with a slash).
\end{funcdesc}

\begin{funcdesc}{isfile}{p}
Return true if \var{p} is an existing regular file.  This follows
symbolic links, so both islink() and isfile() can be true for the same
path.
\end{funcdesc}

\begin{funcdesc}{isdir}{p}
Return true if \var{p} is an existing directory.  This follows
symbolic links, so both islink() and isdir() can be true for the same
path.
\end{funcdesc}

\begin{funcdesc}{islink}{p}
Return true if
\var{p}
refers to a directory entry that is a symbolic link.
Always false if symbolic links are not supported.
\end{funcdesc}

\begin{funcdesc}{ismount}{p}
Return true if \var{p} is a mount point.  (This currently checks whether
\code{\var{p}/..} is on a different device as \var{p} or whether
\code{\var{p}/..} and \var{p} point to the same i-node on the same
device --- is this test correct for all \UNIX{} and POSIX variants?)
\end{funcdesc}

\begin{funcdesc}{join}{p\, q}
Join the paths
\var{p}
and
\var{q} intelligently:
If
\var{q}
is an absolute path, the return value is
\var{q}.
Otherwise, the concatenation of
\var{p}
and
\var{q}
is returned, with a slash (\code{'/'}) inserted unless
\var{p}
is empty or ends in a slash.
\end{funcdesc}

\begin{funcdesc}{normcase}{p}
Normalize the case of a pathname.  This returns the path unchanged;
however, a similar function in \code{macpath} converts upper case to
lower case.
\end{funcdesc}

\begin{funcdesc}{samefile}{p\, q}
Return true if both pathname arguments refer to the same file or directory
(as indicated by device number and i-node number).
Raise an exception if a stat call on either pathname fails.
\end{funcdesc}

\begin{funcdesc}{split}{p}
Split the pathname \var{p} in a pair \code{(\var{head}, \var{tail})}, where
\var{tail} is the last pathname component and \var{head} is
everything leading up to that.  If \var{p} ends in a slash (except if
it is the root), the trailing slash is removed and the operation
applied to the result; otherwise, \code{join(\var{head}, \var{tail})} equals
\var{p}.  The \var{tail} part never contains a slash.  Some boundary
cases: if \var{p} is the root, \var{head} equals \var{p} and
\var{tail} is empty; if \var{p} is empty, both \var{head} and
\var{tail} are empty; if \var{p} contains no slash, \var{head} is
empty and \var{tail} equals \var{p}.
\end{funcdesc}

\begin{funcdesc}{splitext}{p}
Split the pathname \var{p} in a pair \code{(\var{root}, \var{ext})}
such that \code{\var{root} + \var{ext} == \var{p}},
the last component of \var{root} contains no periods,
and \var{ext} is empty or begins with a period.
\end{funcdesc}

\begin{funcdesc}{walk}{p\, visit\, arg}
Calls the function \var{visit} with arguments
\code{(\var{arg}, \var{dirname}, \var{names})} for each directory in the
directory tree rooted at \var{p} (including \var{p} itself, if it is a
directory).  The argument \var{dirname} specifies the visited directory,
the argument \var{names} lists the files in the directory (gotten from
\code{posix.listdir(\var{dirname})}).  The \var{visit} function may
modify \var{names} to influence the set of directories visited below
\var{dirname}, e.g., to avoid visiting certain parts of the tree.  (The
object referred to by \var{names} must be modified in place, using
\code{del} or slice assignment.)
\end{funcdesc}

\section{Standard Module \sectcode{getopt}}

\stmodindex{getopt}
This module helps scripts to parse the command line arguments in
\code{sys.argv}.
It uses the same conventions as the \UNIX{}
\code{getopt()}
function.
It defines the function
\code{getopt.getopt(args, options)}
and the exception
\code{getopt.error}.

The first argument to
\code{getopt()}
is the argument list passed to the script with its first element
chopped off (i.e.,
\code{sys.argv[1:]}).
The second argument is the string of option letters that the
script wants to recognize, with options that require an argument
followed by a colon (i.e., the same format that \UNIX{}
\code{getopt()}
uses).
The return value consists of two elements: the first is a list of
option-and-value pairs; the second is the list of program arguments
left after the option list was stripped (this is a trailing slice of the
first argument).
Each option-and-value pair returned has the option as its first element,
prefixed with a hyphen (e.g.,
\code{'-x'}),
and the option argument as its second element, or an empty string if the
option has no argument.
The options occur in the list in the same order in which they were
found, thus allowing multiple occurrences.
Example:

\bcode\begin{verbatim}
>>> import getopt, string
>>> args = string.split('-a -b -cfoo -d bar a1 a2')
>>> args
['-a', '-b', '-cfoo', '-d', 'bar', 'a1', 'a2']
>>> optlist, args = getopt.getopt(args, 'abc:d:')
>>> optlist
[('-a', ''), ('-b', ''), ('-c', 'foo'), ('-d', 'bar')]
>>> args
['a1', 'a2']
>>> 
\end{verbatim}\ecode

The exception
\code{getopt.error = 'getopt error'}
is raised when an unrecognized option is found in the argument list or
when an option requiring an argument is given none.
The argument to the exception is a string indicating the cause of the
error.

\chapter{UNIX ONLY}

\section{Built-in Module \sectcode{pwd}}

\bimodindex{pwd}
This module provides access to the \UNIX{} password database.
It is available on all \UNIX{} versions.

Password database entries are reported as 7-tuples containing the
following items from the password database (see \file{<pwd.h>}), in order:
\code{pw_name},
\code{pw_passwd},
\code{pw_uid},
\code{pw_gid},
\code{pw_gecos},
\code{pw_dir},
\code{pw_shell}.
The uid and gid items are integers, all others are strings.
An exception is raised if the entry asked for cannot be found.

It defines the following items:

\renewcommand{\indexsubitem}{(in module pwd)}
\begin{funcdesc}{getpwuid}{uid}
Return the password database entry for the given numeric user ID.
\end{funcdesc}

\begin{funcdesc}{getpwnam}{name}
Return the password database entry for the given user name.
\end{funcdesc}

\begin{funcdesc}{getpwall}{}
Return a list of all available password database entries, in arbitrary order.
\end{funcdesc}

\section{Built-in Module \sectcode{grp}}

\bimodindex{grp}
This module provides access to the \UNIX{} group database.
It is available on all \UNIX{} versions.

Group database entries are reported as 4-tuples containing the
following items from the group database (see \file{<grp.h>}), in order:
\code{gr_name},
\code{gr_passwd},
\code{gr_gid},
\code{gr_mem}.
The gid is an integer, name and password are strings, and the member
list is a list of strings.
(Note that most users are not explicitly listed as members of the
group(s) they are in.)
An exception is raised if the entry asked for cannot be found.

It defines the following items:

\renewcommand{\indexsubitem}{(in module grp)}
\begin{funcdesc}{getgrgid}{gid}
Return the group database entry for the given numeric group ID.
\end{funcdesc}

\begin{funcdesc}{getgrnam}{name}
Return the group database entry for the given group name.
\end{funcdesc}

\begin{funcdesc}{getgrall}{}
Return a list of all available group entries entries, in arbitrary order.
\end{funcdesc}

\section{Built-in Module \sectcode{socket}}

\bimodindex{socket}
This module provides access to the BSD {\em socket} interface.
It is available on \UNIX{} systems that support this interface.

For an introduction to socket programming (in C), see the following
papers: \emph{An Introductory 4.3BSD Interprocess Communication
Tutorial}, by Stuart Sechrest and \emph{An Advanced 4.3BSD Interprocess
Communication Tutorial}, by Samuel J.  Leffler et al, both in the
\UNIX{} Programmer's Manual, Supplementary Documents 1 (sections PS1:7
and PS1:8).  The \UNIX{} manual pages for the various socket-related
system calls also a valuable source of information on the details of
socket semantics.

The Python interface is a straightforward transliteration of the
\UNIX{} system call and library interface for sockets to Python's
object-oriented style: the \code{socket()} function returns a
\dfn{socket object} whose methods implement the various socket system
calls.  Parameter types are somewhat higer-level than in the C
interface: as for \code{read()} and \code{write()} operations on Python
files, buffer allocation on receive operations is automatic, and
buffer length is implicit on send operations.

Socket addresses are represented as a single string for the
\code{AF_UNIX} address family and as a pair
\code{(\var{host}, \var{port})} for the \code{AF_INET} address family,
where \var{host} is a string representing
either a hostname in Internet domain notation like
\code{'daring.cwi.nl'} or an IP address like \code{'100.50.200.5'},
and \var{port} is an integral port number.  Other address families are
currently not supported.  The address format required by a particular
socket object is automatically selected based on the address family
specified when the socket object was created.

All errors raise exceptions.  The normal exceptions for invalid
argument types and out-of-memory conditions can be raised; errors
related to socket or address semantics raise the error \code{socket.error}.

Not all socket operations are currently implemented; there are no
provisions for asynchronous or non-blocking I/O (but see
\code{avail()}, and some of the lesser-used primitives such as
\code{getpeername()} are not provided.

The module \code{socket} exports the following constants and functions:

\renewcommand{\indexsubitem}{(in module socket)}
\begin{excdesc}{error}
This exception is raised for socket- or address-related errors.
The accompanying value is either a string telling what went wrong or a
pair \code{(\var{errno}, \var{string})}
representing an error returned by a system
call, similar to the value accompanying \code{posix.error}.
\end{excdesc}

\begin{datadesc}{AF_UNIX}
\dataline{AF_INET}
These constants represent the address (and protocol) families,
used for the first argument to \code{socket()}.
\end{datadesc}

\begin{datadesc}{SOCK_STREAM}
\dataline{SOCK_DGRAM}
These constants represent the socket types,
used for the second argument to \code{socket()}.
(There are other types, but only \code{SOCK_STREAM} and
\code{SOCK_DGRAM} appear to be generally useful.)
\end{datadesc}

\begin{funcdesc}{gethostbyname}{hostname}
Translate a host name to IP address format.  The IP address is
returned as a string, e.g.,  \code{'100.50.200.5'}.  If the host name
is an IP address itself it is returned unchanged.
\end{funcdesc}

\begin{funcdesc}{getservbyname}{servicename\, protocolname}
Translate an Internet service name and protocol name to a port number
for that service.  The protocol name should be \code{'tcp'} or
\code{'udp'}.
\end{funcdesc}

\begin{funcdesc}{socket}{family\, type\, proto}
Create a new socket using the given address family, socket type and
protocol number.  The address family should be \code{AF_INET} or
\code{AF_UNIX}.  The socket type should be \code{SOCK_STREAM},
\code{SOCK_DGRAM} or perhaps one of the other \samp{SOCK_} constants.
The protocol number is usually zero and may be omitted in that case.
\end{funcdesc}

\begin{funcdesc}{fromfd}{fd\, family\, type\, proto}
Build a socket object from an existing file descriptor (an integer as
returned by a file object's \code{fileno} method).  Address family,
socket type and protocol number are as for the \code{socket} function
above.  The file descriptor should refer to a socket, but this is not
checked --- subsequent operations on the object may fail if the file
descriptor is invalid.  This function is rarely needed, but can be
used to get or set socket options on a socket passed to a program as
standard input or output (e.g. a server started by the \UNIX{} inet
daemon).
\end{funcdesc}

\subsection{Socket Object Methods}

\noindent
Socket objects have the following methods.  Except for
\code{makefile()} these correspond to \UNIX{} system calls applicable to
sockets.

\renewcommand{\indexsubitem}{(socket method)}
\begin{funcdesc}{accept}{}
Accept a connection.
The socket must be bound to an address and listening for connections.
The return value is a pair \code{(\var{conn}, \var{address})}
where \var{conn} is a \emph{new} socket object usable to send and
receive data on the connection, and \var{address} is the address bound
to the socket on the other end of the connection.
\end{funcdesc}

\begin{funcdesc}{avail}{}
Return true (nonzero) if at least one byte of data can be received
from the socket without blocking, false (zero) if not.  There is no
indication of how many bytes are available.  (\strong{This function is
obsolete --- see module \code{select} for a more general solution.})
\end{funcdesc}

\begin{funcdesc}{bind}{address}
Bind the socket to an address.  The socket must not already be bound.
\end{funcdesc}

\begin{funcdesc}{close}{}
Close the socket.  All future operations on the socket object will fail.
The remote end will receive no more data (after queued data is flushed).
Sockets are automatically closed when they are garbage-collected.
\end{funcdesc}

\begin{funcdesc}{connect}{address}
Connect to a remote socket.
\end{funcdesc}

\begin{funcdesc}{fileno}{}
Return the socket's file descriptor (a small integer).  This is useful
with \code{select}.
\end{funcdesc}

\begin{funcdesc}{getpeername}{}
Return the remote address to which the socket is connected.  This is
useful to find out the port number of a remote IP socket, for instance.
\end{funcdesc}

\begin{funcdesc}{getsockname}{}
Return the socket's own address.  This is useful to find out the port
number of an IP socket, for instance.
\end{funcdesc}

\begin{funcdesc}{getsockopt}{level\, optname\, buflen}
Return the value of the given socket option (see the \UNIX{} man page
{\it getsockopt}(2)).  The needed symbolic constants are defined in module
SOCKET.  If the optional third argument is absent, an integer option
is assumed and its integer value is returned by the function.  If
\var{buflen} is present, it specifies the maximum length of the buffer used
to receive the option in, and this buffer is returned as a string.
It's up to the caller to decode the contents of the buffer (see the
optional built-in module \code{struct} for a way to decode C structures
encoded as strings).
\end{funcdesc}

\begin{funcdesc}{listen}{backlog}
Listen for connections made to the socket.
The argument (in the range 0-5) specifies the maximum number of
queued connections.
\end{funcdesc}

\begin{funcdesc}{makefile}{mode}
Return a \dfn{file object} associated with the socket.
(File objects were described earlier under Built-in Types.)
The file object references a \code{dup}ped version of the socket file
descriptor, so the file object and socket object may be closed or
garbage-collected independently.
\end{funcdesc}

\begin{funcdesc}{recv}{bufsize\, flags}
Receive data from the socket.  The return value is a string representing
the data received.  The maximum amount of data to be received
at once is specified by \var{bufsize}.  See the \UNIX{} manual page
for the meaning of the optional argument \var{flags}; it defaults to
zero.
\end{funcdesc}

\begin{funcdesc}{recvfrom}{bufsize}
Receive data from the socket.  The return value is a pair
\code{(\var{string}, \var{address})} where \var{string} is a string
representing the data received and \var{address} is the address of the
socket sending the data.
\end{funcdesc}

\begin{funcdesc}{send}{string}
Send data to the socket.  The socket must be connected to a remote
socket.
\end{funcdesc}

\begin{funcdesc}{sendto}{string\, address}
Send data to the socket.  The socket should not be connected to a
remote socket, since the destination socket is specified by
\code{address}.
\end{funcdesc}

\begin{funcdesc}{setsockopt}{level\, optname\, value}
Set the value of the given socket option (see the \UNIX{} man page
{\it setsockopt}(2)).  The needed symbolic constants are defined in module
\code{SOCKET}.  The value can be an integer or a string representing a
buffer.  In the latter case it is up to the caller to ensure that the
string contains the proper bits (see the optional built-in module
\code{struct} for a way to encode C structures as strings).
\end{funcdesc}

\begin{funcdesc}{shutdown}{how}
Shut down one or both halves of the connection.  If \var{how} is \code{0},
further receives are disallowed.  If \var{how} is \code{1}, further sends are
disallowed.  If \var{how} is \code{2}, further sends and receives are
disallowed.
\end{funcdesc}

Note that there are no methods \code{read()} or \code{write()}; use
\code{recv()} and \code{send()} without \var{flags} argument instead.

\subsection{Example}
\nodename{Socket Example}

Here are two minimal example programs using the TCP/IP protocol: a
server that echoes all data that it receives back (servicing only one
client), and a client using it.  Note that a server must perform the
sequence \code{socket}, \code{bind}, \code{listen}, \code{accept}
(possibly repeating the \code{accept} to service more than one client),
while a client only needs the sequence \code{socket}, \code{connect}.
Also note that the server does not \code{send}/\code{receive} on the
socket it is listening on but on the new socket returned by
\code{accept}.

\bcode\begin{verbatim}
# Echo server program
from socket import *
HOST = ''                 # Symbolic name meaning the local host
PORT = 50007              # Arbitrary non-privileged server
s = socket(AF_INET, SOCK_STREAM)
s.bind(HOST, PORT)
s.listen(0)
conn, addr = s.accept()
print 'Connected by', addr
while 1:
    data = conn.recv(1024)
    if not data: break
    conn.send(data)
conn.close()
\end{verbatim}\ecode

\bcode\begin{verbatim}
# Echo client program
from socket import *
HOST = 'daring.cwi.nl'    # The remote host
PORT = 50007              # The same port as used by the server
s = socket(AF_INET, SOCK_STREAM)
s.connect(HOST, PORT)
s.send('Hello, world')
data = s.recv(1024)
s.close()
print 'Received', `data`
\end{verbatim}\ecode

\section{Built-in module \sectcode{select}}

This module provides access to the function \code{select} available in
most \UNIX{} versions.  It defines the following:

\renewcommand{\indexsubitem}{(in module select)}
\begin{excdesc}{error}
The exception raised when an error occurs.  The accompanying value is
a pair containing the numeric error code from \code{errno} and the
corresponding string, as would be printed by the C function
\code{perror()}.
\end{excdesc}

\begin{funcdesc}{select}{iwtd\, owtd\, ewtd\, timeout}
This is a straightforward interface to the \UNIX{} \code{select()}
system call.  The first three arguments are lists of `waitable
objects': either integers representing \UNIX{} file descriptors or
objects with a parameterless method named \code{fileno()} returning
such an integer.  The three lists of waitable objects are for input,
output and `exceptional conditions', respectively.  Empty lists are
allowed.  The optional last argument is a time-out specified as a
floating point number in seconds.  When the \var{timeout} argument
is omitted the function blocks until at least one file descriptor is
ready.  A time-out value of zero specifies a poll and never blocks.

The return value is a triple of lists of objects that are ready:
subsets of the first three arguments.  When the time-out is reached
without a file descriptor becoming ready, three empty lists are
returned.

Amongst the acceptable object types in the lists are Python file
objects (e.g. \code{sys.stdin}, or objects returned by \code{open()}
or \code{posix.popen()}), socket objects returned by
\code{socket.socket()}, and the module \code{stdwin} which happens to
define a function \code{fileno()} for just this purpose.  You may
also define a \dfn{wrapper} class yourself, as long as it has an
appropriate \code{fileno()} method (that really returns a \UNIX{} file
descriptor, not just a random integer).
\end{funcdesc}
\bimodindex{socket}
\bimodindex{stdwin}

\section{Built-in Module \sectcode{dbm}}

Dbm provides python programs with an interface to the unix \code{ndbm}
database library. Dbm objects are of the mapping type, so they can be
handled just like objects of the built-in \dfn{dictionary} type. Keys
are always strings, like with dictionary objects, but in contrast to
dictionaries the values stored in a dbm object should also all be of
string type. The only other difference with dictionaries is that dbm
objects cannot be printed, for obvious reasons.

The module defines the following constant and functions:

\renewcommand{\indexsubitem}{(in module dbm)}
\begin{excdesc}{error}
Raised on dbm-specific errors, such as I/O errors. \code{KeyError} is
raised for general mapping errors like specifying an incorrect key.
\end{excdesc}

\begin{funcdesc}{open}{filename\, rwmode\, filemode}
Open a dbm database and return a mapping object.  \var{filename} is
the name of the database file (without the \file{.dir} or \file{.pag}
extensions), \var{rwmode} is \code{'r'}, \code{'w'} or \code{'rw'} as for
\code{open}, and \var{filemode} is the unix mode of the file, used only
when the database has to be created.
\end{funcdesc}

\section{Built-in Module \sectcode{thread}}

This module provides low-level primitives for working with multiple
threads (a.k.a. \dfn{light-weight processes} or \dfn{tasks}) --- multiple
threads of control sharing their global data space.  For
synchronization, simple locks (a.k.a. \dfn{mutexes} or \dfn{binary
semaphores}) are provided.

The module is optional and supported on SGI and Sun Sparc systems only.

It defines the following constant and functions:

\renewcommand{\indexsubitem}{(in module thread)}
\begin{excdesc}{error}
Raised on thread-specific errors.
\end{excdesc}

\begin{funcdesc}{start_new_thread}{func\, arg}
Start a new thread.  The thread executes the function \var{func}
with the argument list \var{arg} (which must be a tuple).  When the
function returns, the thread silently exits.  When the function raises
terminates with an unhandled exception, a stack trace is printed and
then the thread exits (but other threads continue to run).
\end{funcdesc}

\begin{funcdesc}{exit_thread}{}
Exit the current thread silently.  Other threads continue to run.
\strong{Caveat:} code in pending \code{finally} clauses is not executed.
\end{funcdesc}

\begin{funcdesc}{exit_prog}{status}
Exit all threads and report the value of the integer argument
\var{status} as the exit status of the entire program.
\strong{Caveat:} code in pending \code{finally} clauses, in this thread
or in other threads, is not executed.
\end{funcdesc}

\begin{funcdesc}{allocate_lock}{}
Return a new lock object.  Methods of locks are described below.  The
lock is initially unlocked.
\end{funcdesc}

Lock objects have the following methods:

\renewcommand{\indexsubitem}{(lock method)}
\begin{funcdesc}{acquire}{waitflag}
Without the optional argument, this method acquires the lock
unconditionally, if necessary waiting until it is released by another
thread (only one thread at a time can acquire a lock --- that's their
reason for existence), and returns \code{None}.  If the integer
\var{waitflag} argument is present, the action depends on its value:
if it is zero, the lock is only acquired if it can be acquired
immediately without waiting, while if it is nonzero, the lock is
acquired unconditionally as before.  If an argument is present, the
return value is 1 if the lock is acquired successfully, 0 if not.
\end{funcdesc}

\begin{funcdesc}{release}{}
Releases the lock.  The lock must have been acquired earlier, but not
necessarily by the same thread.
\end{funcdesc}

\begin{funcdesc}{locked}{}
Return the status of the lock: 1 if it has been acquired by some
thread, 0 if not.
\end{funcdesc}

{\bf Caveats:}

\begin{itemize}
\item
Threads interact strangely with interrupts: the
\code{KeyboardInterrupt} exception will be received by an arbitrary
thread.

\item
Calling \code{sys.exit(\var{status})} or executing
\code{raise SystemExit, \var{status}} is almost equivalent to calling
\code{thread.exit_prog(\var{status})}, except that the former ways of
exiting the entire program do honor \code{finally} clauses in the
current thread (but not in other threads).

\item
Not all built-in functions that may block waiting for I/O allow other
threads to run, although the most popular ones (\code{sleep},
\code{read}, \code{select}) work as expected.

\end{itemize}

\chapter{AMOEBA ONLY}

\section{Built-in Module \sectcode{amoeba}}

\bimodindex{amoeba}
This module provides some object types and operations useful for
Amoeba applications.  It is only available on systems that support
Amoeba operations.  RPC errors and other Amoeba errors are reported as
the exception \code{amoeba.error = 'amoeba.error'}.

The module \code{amoeba} defines the following items:

\renewcommand{\indexsubitem}{(in module amoeba)}
\begin{funcdesc}{name_append}{path\, cap}
Stores a capability in the Amoeba directory tree.
Arguments are the pathname (a string) and the capability (a capability
object as returned by
\code{name_lookup()}).
\end{funcdesc}

\begin{funcdesc}{name_delete}{path}
Deletes a capability from the Amoeba directory tree.
Argument is the pathname.
\end{funcdesc}

\begin{funcdesc}{name_lookup}{path}
Looks up a capability.
Argument is the pathname.
Returns a
\dfn{capability}
object, to which various interesting operations apply, described below.
\end{funcdesc}

\begin{funcdesc}{name_replace}{path\, cap}
Replaces a capability in the Amoeba directory tree.
Arguments are the pathname and the new capability.
(This differs from
\code{name_append()}
in the behavior when the pathname already exists:
\code{name_append()}
finds this an error while
\code{name_replace()}
allows it, as its name suggests.)
\end{funcdesc}

\begin{datadesc}{capv}
A table representing the capability environment at the time the
interpreter was started.
(Alas, modifying this table does not affect the capability environment
of the interpreter.)
For example,
\code{amoeba.capv['ROOT']}
is the capability of your root directory, similar to
\code{getcap("ROOT")}
in C.
\end{datadesc}

\begin{excdesc}{error}
The exception raised when an Amoeba function returns an error.
The value accompanying this exception is a pair containing the numeric
error code and the corresponding string, as returned by the C function
\code{err_why()}.
\end{excdesc}

\begin{funcdesc}{timeout}{msecs}
Sets the transaction timeout, in milliseconds.
Returns the previous timeout.
Initially, the timeout is set to 2 seconds by the Python interpreter.
\end{funcdesc}

\subsection{Capability Operations}

Capabilities are written in a convenient ASCII format, also used by the
Amoeba utilities
{\it c2a}(U)
and
{\it a2c}(U).
For example:

\bcode\begin{verbatim}
>>> amoeba.name_lookup('/profile/cap')
aa:1c:95:52:6a:fa/14(ff)/8e:ba:5b:8:11:1a
>>> 
\end{verbatim}\ecode

The following methods are defined for capability objects.

\renewcommand{\indexsubitem}{(capability method)}
\begin{funcdesc}{dir_list}{}
Returns a list of the names of the entries in an Amoeba directory.
\end{funcdesc}

\begin{funcdesc}{b_read}{offset\, maxsize}
Reads (at most)
\var{maxsize}
bytes from a bullet file at offset
\var{offset.}
The data is returned as a string.
EOF is reported as an empty string.
\end{funcdesc}

\begin{funcdesc}{b_size}{}
Returns the size of a bullet file.
\end{funcdesc}

\begin{funcdesc}{dir_append}{}
\funcline{dir_delete}{}\ 
\funcline{dir_lookup}{}\ 
\funcline{dir_replace}{}
Like the corresponding
\samp{name_}*
functions, but with a path relative to the capability.
(For paths beginning with a slash the capability is ignored, since this
is the defined semantics for Amoeba.)
\end{funcdesc}

\begin{funcdesc}{std_info}{}
Returns the standard info string of the object.
\end{funcdesc}

\begin{funcdesc}{tod_gettime}{}
Returns the time (in seconds since the Epoch, in UCT, as for POSIX) from
a time server.
\end{funcdesc}

\begin{funcdesc}{tod_settime}{t}
Sets the time kept by a time server.
\end{funcdesc}

\chapter{MACINTOSH ONLY}

The following modules are available on the Apple Macintosh only.

\section{Built-in module \sectcode{mac}}

\bimodindex{mac}
This module provides a subset of the operating system dependent
functionality provided by the optional built-in module \code{posix}.
It is best accessed through the more portable standard module
\code{os}.

The following functions are available in this module:
\code{chdir},
\code{getcwd},
\code{listdir},
\code{mkdir},
\code{rename},
\code{rmdir},
\code{stat},
\code{sync},
\code{unlink},
as well as the exception \code{error}.

\section{Standard module \sectcode{macpath}}

\stmodindex{macpath}
This module provides a subset of the pathname manipulation functions
available from the optional standard module \code{posixpath}.  It is
best accessed through the more portable standard module \code{os}, as
\code{os.path}.

The following functions are available in this module:
\code{normcase},
\code{isabs},
\code{join},
\code{split},
\code{isdir},
\code{isfile},
\code{exists}.
